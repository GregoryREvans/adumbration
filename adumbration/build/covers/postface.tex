\documentclass[11pt]{article}
\usepackage{fontspec}
\usepackage[utf8]{inputenc}
\setmainfont{STIXGeneral}
\usepackage[paperwidth=8.5in,paperheight=11in,margin=1in,headheight=0.0in,footskip=0.5in,includehead,includefoot,portrait]{geometry}
\usepackage[absolute]{textpos}
\TPGrid[0.5in, 0.25in]{23}{24}
\parindent=0pt
\parskip=12pt
\usepackage{nopageno}
\usepackage{graphicx}
\graphicspath{ {./images/} }
\usepackage{amsmath}
\usepackage{tikz}
\newcommand*\circled[1]{\tikz[baseline=(char.base)]{
            \node[shape=circle,draw,inner sep=1pt] (char) {#1};}}

\begin{document}

\begin{textblock}{23}(0, 1)
\begin{center}
\huge AFTERWORD
\end{center}
\end{textblock}

\begingroup
\begin{center}
\leftskip0.4in
To become imperceptible oneself, to have dismantled love in order to become capable of loving. To have dismantled one's self in order finally to be alone and meet the true double at the other end of the line [...] but this, precisely, is a becoming only for one who knows how to be nobody, to no longer be anybody. To paint oneself gray on gray.\color{white}..............................
\rightskip\leftskip
\phantom{text} \hfill \color{black}(Gilles Deleuze, \textit{A Thousand Plateaus: Capitalism and Schizophrenia})
\end{center}

\begin{center}
\leftskip0.4in
The shadow escapes from the body like an animal we had been sheltering.\color{white}..............................
\rightskip\leftskip
\phantom{text} \hfill \color{black}(Gilles Deleuze, \textit{Francis Bacon: The Logic of Sensation})
\end{center}

\begin{center}
\leftskip0.4in
For what is this shadow of the going in which we come, this shadow of the coming in which we go, this shadow of the coming and the going in which we wait, if not the shadow of purpose, of the purpose that budding withers, that withering buds, whose blooming is a budding withering.
\rightskip\leftskip
\phantom{text} \hfill (Samuel Beckett, \textit{Watt})
\end{center}

\begin{center}
\leftskip0.4in
[...] Cut my shadow from me. \\
Free me from the torment \\
of seeing myself without fruit. \newline
[...] The day walks in circles around me, \\
and the night copies me \\
in all its stars. [...] \color{white}..........................................................................................................................
\rightskip\leftskip
\phantom{text} \hfill \color{black}(Federico Garc\'ia Lorca, \textit{Song of the Barren Orange Tree})
\end{center}

\begin{center}
\leftskip0.4in
[...] so I love you because I know no other way \\
than this: where I does not exist, nor you, \\
so close that your hand on my chest is my hand, \\
so close that your eyes close as I fall asleep \color{white}..............................................................................
\rightskip\leftskip
\phantom{text} \hfill \color{black}(Pablo Neruda, \textit{Between the Shadow and the Soul})
\end{center}

\begin{center}
\leftskip0.4in
In the resonance we hear the poem, in the reverberations we speak it, it is our own. \color{white}..................................................
\rightskip\leftskip
\phantom{text} \hfill \color{black}(Bachelard Gaston, \textit{The Poetics of Space})
\end{center}

\begin{center}
\leftskip0.4in
When the iron shadows hunker down, you unload the little boat of your sorrow and we climb aboard.
\rightskip\leftskip
\phantom{text} \hfill (Andrew Grace, \textit{Sancta})
\end{center}

\begin{center}
\leftskip0.4in
It is highly unlikely that we, who can know, determine, and define the natural essences of all things surrounding us, which we are not, should ever be able to do the same for ourselves -- this would be like jumping over our own shadows.
\rightskip\leftskip
\phantom{text} \hfill (Hannah Arendt, \textit{The Human Condition})
\end{center}
\endgroup

\end{document}
